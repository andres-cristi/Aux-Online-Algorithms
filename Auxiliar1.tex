\documentclass[11pt, spanish]{article}
\pagestyle{empty}

%acentos de la forma � en vez de \'a
\usepackage[spanish]{babel}
\usepackage[utf8]{inputenc}

%Enumeración con columnas que se usa con \begin{multicols}{2} \begin{enumerate}
\usepackage{multicol}
\usepackage{multirow}


%Para poder usar begin comment
\usepackage{verbatim}

%Teoremas
\usepackage{amsthm}
\theoremstyle{plain}
\newtheorem{teo}{Teorema}


%letras para enumerar
\makeatletter\renewcommand\theenumi{\@alph\c@enumi}\makeatother\renewcommand\labelenumi{\theenumi)}

%m�rgenes
\usepackage[left=2cm,top=2cm,right=2cm]{geometry}

\usepackage{fancyhdr}
\pagestyle{fancy}
\usepackage{enumerate}

%conjuntos N, Q, R, Z, C
\usepackage{dsfont}
\newcommand{\N}{\mathds{N}}
\newcommand{\Q}{\mathds{Q}}
\newcommand{\R}{\mathds{R}}
\newcommand{\Z}{\mathds{Z}}
\newcommand{\C}{\mathds{C}}
\newcommand{\M}{\mathcal{M}}
\newcommand{\Pol}{\mathcal{P}}

%Transformada de Laplace
\renewcommand{\L}{\mathcal{L}}

%Probabilidades
\newcommand{\PP}{\mathbb{P}}
\newcommand{\E}{\mathbb{E}}
\newcommand{\B}{\mathcal{B}}
\newcommand{\Var}{\mathbb{V}\text{ar}}

%l�gica
\newcommand{\ssi}{\Leftrightarrow}

%funciones
\newcommand{\function}[5]{  \begin{array}{rl}
                                #1: #2 &\longrightarrow #3 \\
                                #4 & \longmapsto #1\left(#4\right)= #5
                            \end{array} }

\newcommand{\funcion}[3]{#1: #2 \longrightarrow #3 }
\newcommand{\parteentera}[1]{[#1]}

%%%operadores matematicos
\usepackage{mathtools}
\DeclarePairedDelimiter\abs{\lvert}{\rvert}%
\DeclarePairedDelimiter\norm{\lVert}{\rVert}%
\newcommand{\tr}{\text{tr}}


% Swap the definition of \abs* and \norm*, so that \abs
% and \norm resizes the size of the brackets, and the 
% starred version does not.
\makeatletter
\let\oldabs\abs
\def\abs{\@ifstar{\oldabs}{\oldabs*}}
%
\let\oldnorm\norm
\def\norm{\@ifstar{\oldnorm}{\oldnorm*}}
\makeatother
% % % % % % % % % %
%\providecommand{\abs}[1]{\lvert#1 \rvert}
%\providecommand{\norm}[1]{\lVert#1 \rVert}
%\providecommand{\pin}[2]{\left< #1,#2 \right>} %producto interno

\providecommand{\dpartial}[2]{\frac{\partial #1}{\partial #2}} %derivada parcial

\usepackage{amssymb}
\usepackage{amsmath, amsthm, amsfonts}

%Im�genes
\usepackage{graphicx}
\usepackage{float}
\begin{document}
\fancyhead[L]{Facultad de Ciencias Físicas y Matemáticas}
\fancyhead[R]{Universidad de Chile}

\begin{flushleft}
  \textbf{Online Algorithms and Scheduling}
  \\\textbf{Profesor:} Andreas Wiese.
  \\\textbf{Auxiliar:} Andrés Cristi.
\end{flushleft}


\begin{center}
  \large{\textbf{Clase Auxiliar 1\\ Multislope Ski Rental Problem}}
\end{center}

%\begin{flushleft}



\begin{itemize}
\item[\textbf{P1.}] Consideramos el \emph{Ski Rental Problem}, pero
con la variante de que hay acciones intermedias entre arrendar y comprar:
se puede comprar parcialmente, con lo que se accede a una renta menor.

Representamos lo anterior con los valores $b_i,r_i$ para $i=0,\cdots, k$,
como los valores de compra y renta en los estados $i$. El estado inicial
tiene precio de compra $b_0=0$ y el final tiene precio de renta $r_k=0$.
Suponemos que los $b_i$ son crecientes y los $r_i$ decrecientes. Cada nuevo día de ski se debe pagar $r_i$, donde $i$ es el estado actual. Se
puede pasar del estado $i$ al $i+1$ pagando $b_{i+1}-b_i$.

\begin{enumerate}
    \item Para $i=1,\cdots,k$ defina $k$ distintas instancias del
    \emph{Ski Rental Problem} clásico, dadas por
    \[
     \beta_i= b_i - b_{i-1}; \; \; \rho_i = r_{i-1} - r_i
    \]
    Denote por $OPT(t)$ y $OPT_i(t)$ las soluciones óptimas offline
    del problema original y de las instancias recién definidas cuando
    se esquía $t$ días. Pruebe que para todo $t$
    \[
      OPT(t) = \sum_{i=1}^k OPT_i(t)
    \]
    
    \item Sea $A$ un algoritmo aleatorizado para el problema clásico,
    y sea $p_i(t)$ la probabilidad de que $A$ compre en $[0,t]$ en la
    instancia $(\beta_i, \rho_i)$. Suponga que $A$ es tal que $p_i(t)\geq p_{i+1}(t)$ para todo $t,i$ (explique por qué es un supuesto
    razonable) y defina $\hat{p}_i(t)= p_i(t)-p_{i+1}(t)$
    para $i=1,...,k-1$,
    $\hat{p}_0(t)=1-p_1(t)$ y $\hat{p}_k(t)= p_k(t)$. Pruebe que
    para todo $t$ se cumple que
    $\sum_{i=0}^k \hat{p}_i(t) =1$ y construya un algoritmo aleatorizado
    $B$ para el problema original tal que la probabilidad de que
    esté en el estado $i$ en el día $t$ sea igual a $\hat{p}_i(t)$.
    
    \item Muestre que si $A$ es $c$-competitivo para el problema
    clásico, entonces $B$ es $c$-competitivo para la versión
    \emph{multislope}.
\end{enumerate}
\end{itemize}

\end{document} 
