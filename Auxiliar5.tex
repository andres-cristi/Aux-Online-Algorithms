\documentclass[11pt, spanish]{article}
\pagestyle{empty}

%acentos de la forma � en vez de \'a
\usepackage[spanish]{babel}
\usepackage[utf8]{inputenc}

%Enumeración con columnas que se usa con \begin{multicols}{2} \begin{enumerate}
\usepackage{multicol}
\usepackage{multirow}


%Para poder usar begin comment
\usepackage{verbatim}

%Teoremas
\usepackage{amsthm}
\theoremstyle{plain}
\newtheorem{teo}{Teorema}


%letras para enumerar
\makeatletter\renewcommand\theenumi{\@alph\c@enumi}\makeatother\renewcommand\labelenumi{\theenumi)}

%m�rgenes
\usepackage[left=2cm,top=2cm,right=2cm]{geometry}

\usepackage{fancyhdr}
\pagestyle{fancy}
\usepackage{enumerate}

%conjuntos N, Q, R, Z, C
\usepackage{dsfont}
\newcommand{\N}{\mathds{N}}
\newcommand{\Q}{\mathds{Q}}
\newcommand{\R}{\mathds{R}}
\newcommand{\Z}{\mathds{Z}}
\newcommand{\C}{\mathds{C}}
\newcommand{\M}{\mathcal{M}}
\newcommand{\Pol}{\mathcal{P}}

%Transformada de Laplace
\renewcommand{\L}{\mathcal{L}}

%Probabilidades
\newcommand{\PP}{\mathbb{P}}
\newcommand{\E}{\mathbb{E}}
\newcommand{\B}{\mathcal{B}}
\newcommand{\Var}{\mathbb{V}\text{ar}}

%l�gica
\newcommand{\ssi}{\Leftrightarrow}

%funciones
\newcommand{\function}[5]{  \begin{array}{rl}
                                #1: #2 &\longrightarrow #3 \\
                                #4 & \longmapsto #1\left(#4\right)= #5
                            \end{array} }

\newcommand{\funcion}[3]{#1: #2 \longrightarrow #3 }
\newcommand{\parteentera}[1]{[#1]}

%%%operadores matematicos
\usepackage{mathtools}
\DeclarePairedDelimiter\abs{\lvert}{\rvert}%
\DeclarePairedDelimiter\norm{\lVert}{\rVert}%
\newcommand{\tr}{\text{tr}}


% Swap the definition of \abs* and \norm*, so that \abs
% and \norm resizes the size of the brackets, and the 
% starred version does not.
\makeatletter
\let\oldabs\abs
\def\abs{\@ifstar{\oldabs}{\oldabs*}}
%
\let\oldnorm\norm
\def\norm{\@ifstar{\oldnorm}{\oldnorm*}}
\makeatother
% % % % % % % % % %
%\providecommand{\abs}[1]{\lvert#1 \rvert}
%\providecommand{\norm}[1]{\lVert#1 \rVert}
%\providecommand{\pin}[2]{\left< #1,#2 \right>} %producto interno

\providecommand{\dpartial}[2]{\frac{\partial #1}{\partial #2}} %derivada parcial

\usepackage{amssymb}
\usepackage{amsmath, amsthm, amsfonts}

%Im�genes
\usepackage{graphicx}
\usepackage{float}



\newcommand{\bit}{\textsc{bit}}
\newcommand{\opt}{\textsc{opt}}


\begin{document}
\fancyhead[L]{Facultad de Ciencias Físicas y Matemáticas}
\fancyhead[R]{Universidad de Chile}

\begin{flushleft}
  \textbf{Online Algorithms and Scheduling}
  \\\textbf{Profesor:} Andreas Wiese.
  \\\textbf{Auxiliar:} Andrés Cristi.
\end{flushleft}


\begin{center}
  \large{\textbf{Clase Auxiliar 5\\ 16 de Abril }}
\end{center}

%\begin{flushleft}



\begin{itemize}
  \item[\textbf{P1.}] \textbf{Uniform MTS}

    Considere un MTS uniforme de $n$ estados, es decir, un espacio $(X,d)$
    tal que $d(x,y)=1$ para todo $x,y\in X$; en el que las tareas
    toman valores sólo en $\{0,1\}$. Estudiaremos un algoritmo online
    aleatorizado óptimo.
    \begin{enumerate}
      \item Para la cota inferior, considere una instancia de $k$
	tareas, cada una valiendo 1 en un estado aleatorio uniforme y 0
	en el resto, todas independientes. Muestre que el costo
	esperado de cualquier algoritmo determinista es $k/n$.
      \item Sea $c(k)$ el costo esperado del siguiente algoritmo
      offline: en el paso $i$, tomar $a(i)$ el primer paso despu\'es de
      $i$ en el que aparecen todas las tareas asociadas a cada estado;
      si el algoritmo estaba en el estado asociado a $\tau_i$,
      cambiar al estado asociado a $\tau_{a(i)}$, si no, quedarse en
      el mismo. Pruebe que $\lim_k \frac{c(k)}{k}= \frac{1}{n H_n}$.
    \item Concluya que ningún algoritmo aleatorizado es mejor que 
      $H_n$-competitivo.
    \item Para la cota inferior, estudie la forma de la función 
      \textit{offset} del sistema y los costos óptimos de
      transitar entre sus posibles valores.
    \item Considere un algoritmo que toma un estado aleatorio uniforme
      entre los óptimos según la función \textit{offset}. ¿Cuál
      sería la forma razonable de realizar las transiciones?
    \item Tomando el potencial $\Phi=H_m$, donde $m$ es el número de 
      estados óptimos según la función \textit{offset}, pruebe
      que el algoritmo es $H_n$-competitivo.
  \end{enumerate}
\end{itemize}

\end{document} 
