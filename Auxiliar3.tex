\documentclass[11pt, spanish]{article}
\pagestyle{empty}

%acentos de la forma � en vez de \'a
\usepackage[spanish]{babel}
\usepackage[utf8]{inputenc}

%Enumeración con columnas que se usa con \begin{multicols}{2} \begin{enumerate}
\usepackage{multicol}
\usepackage{multirow}


%Para poder usar begin comment
\usepackage{verbatim}

%Teoremas
\usepackage{amsthm}
\theoremstyle{plain}
\newtheorem{teo}{Teorema}


%letras para enumerar
\makeatletter\renewcommand\theenumi{\@alph\c@enumi}\makeatother\renewcommand\labelenumi{\theenumi)}

%m�rgenes
\usepackage[left=2cm,top=2cm,right=2cm]{geometry}

\usepackage{fancyhdr}
\pagestyle{fancy}
\usepackage{enumerate}

%conjuntos N, Q, R, Z, C
\usepackage{dsfont}
\newcommand{\N}{\mathds{N}}
\newcommand{\Q}{\mathds{Q}}
\newcommand{\R}{\mathds{R}}
\newcommand{\Z}{\mathds{Z}}
\newcommand{\C}{\mathds{C}}
\newcommand{\M}{\mathcal{M}}
\newcommand{\Pol}{\mathcal{P}}

%Transformada de Laplace
\renewcommand{\L}{\mathcal{L}}

%Probabilidades
\newcommand{\PP}{\mathbb{P}}
\newcommand{\E}{\mathbb{E}}
\newcommand{\B}{\mathcal{B}}
\newcommand{\Var}{\mathbb{V}\text{ar}}

%l�gica
\newcommand{\ssi}{\Leftrightarrow}

%funciones
\newcommand{\function}[5]{  \begin{array}{rl}
                                #1: #2 &\longrightarrow #3 \\
                                #4 & \longmapsto #1\left(#4\right)= #5
                            \end{array} }

\newcommand{\funcion}[3]{#1: #2 \longrightarrow #3 }
\newcommand{\parteentera}[1]{[#1]}

%%%operadores matematicos
\usepackage{mathtools}
\DeclarePairedDelimiter\abs{\lvert}{\rvert}%
\DeclarePairedDelimiter\norm{\lVert}{\rVert}%
\newcommand{\tr}{\text{tr}}


% Swap the definition of \abs* and \norm*, so that \abs
% and \norm resizes the size of the brackets, and the 
% starred version does not.
\makeatletter
\let\oldabs\abs
\def\abs{\@ifstar{\oldabs}{\oldabs*}}
%
\let\oldnorm\norm
\def\norm{\@ifstar{\oldnorm}{\oldnorm*}}
\makeatother
% % % % % % % % % %
%\providecommand{\abs}[1]{\lvert#1 \rvert}
%\providecommand{\norm}[1]{\lVert#1 \rVert}
%\providecommand{\pin}[2]{\left< #1,#2 \right>} %producto interno

\providecommand{\dpartial}[2]{\frac{\partial #1}{\partial #2}} %derivada parcial

\usepackage{amssymb}
\usepackage{amsmath, amsthm, amsfonts}

%Im�genes
\usepackage{graphicx}
\usepackage{float}



\newcommand{\bit}{\textsc{bit}}
\newcommand{\opt}{\textsc{opt}}


\begin{document}
\fancyhead[L]{Facultad de Ciencias Físicas y Matemáticas}
\fancyhead[R]{Universidad de Chile}

\begin{flushleft}
  \textbf{Online Algorithms and Scheduling}
  \\\textbf{Profesor:} Andreas Wiese.
  \\\textbf{Auxiliar:} Andrés Cristi.
\end{flushleft}


\begin{center}
  \large{\textbf{Clase Auxiliar 3\\ 5 de Abril }}
\end{center}

%\begin{flushleft}



\begin{itemize}
  \item[\textbf{P1.}] \textbf{Online Bipartite Matching}

    Dado un grafo bipartito $G=(U,V,E)$ que posee un \textit{matching} perfecto $M^*$,
    los v\'ertices de $U$ son revelados
    uno a uno de manera on-line, junto con sus aristas. Cada vez que llega un
    nuevo v\'ertice $u$, debe decidirse de manera irrevocable con qu\'e
    elemento de $N(u)$ se va a conectar. El objetivo es maximizar el tamaño
    del \textit{matching} resultante.
    \begin{enumerate}
      \item Pruebe que el algoritmo que a la llegada de $u$, lo asigna a un
	elemento arbitrario de $N(u)$ si es posible es 2-competitivo.
      \item Muestre que cualquier algoritmo determinista es en el mejor
	caso 2-competitivo.

      \item Muestre que $4/3$ es una cota inferior para la competitividad
	de un algoritmo aleatorizado.

      \item Considere el algoritmo \textsc{Ranking}: al comienzo se elige un orden al
	azar $\sigma$ sobre $V$. Cada vez que llega $u$, se elige el primer
	elemento de $N(u)$ disponible según $\sigma$. Pruebe que si $u$ no
	es asignado a $v=M^*(u)$ por \textsc{Ranking}, entonces es asignado a
	un v\'ertice $v'$ con $\sigma(v') \leq \sigma(v)$.

      \item Sea $u\in U$ y $v= M^*(u)$, y considere un orden $\sigma'$. Sea
	$\sigma_i$ la permutación que resulta de mover $v$ a la posición $i$.
	Muestre que si $v$ no es asignado por \textsc{Ranking}$(\sigma')$, entonces,
	para todo $i$, $u$ es asignado por \textsc{Ranking}$(\sigma_i)$ a un
	v\'ertice $v_i$ con $\sigma_i(v_i) \leq \sigma'(v)$.

      \item Sea $x_t$ la probabilidad (dada por la distribución de $\sigma$) de que
	el v\'ertice de $V$ que tiene la posición $t$ sea asignado por
	\textsc{Ranking}. Pruebe que $1-x_t \leq \frac{1}{n} \sum_{1\leq s\leq t} x_s$.

      \item Muestre que en esperanza \textsc{Ranking} asigna a al menos 
	\begin{align*}
	  \sum_{s=1}^n \left( 1- \frac{1}{n+1} \right)^s
	\end{align*}
	parejas.

      \item Concluya que \textsc{Ranking} es $\frac{e}{e-1}$ competitivo.
    \end{enumerate}

\end{itemize}

\end{document} 
