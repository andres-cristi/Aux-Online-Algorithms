\documentclass[11pt, spanish]{article}
\pagestyle{empty}

%acentos de la forma � en vez de \'a
\usepackage[spanish]{babel}
\usepackage[utf8]{inputenc}

%Enumeración con columnas que se usa con \begin{multicols}{2} \begin{enumerate}
\usepackage{multicol}
\usepackage{multirow}


%Para poder usar begin comment
\usepackage{verbatim}

%Teoremas
\usepackage{amsthm}
\theoremstyle{plain}
\newtheorem{teo}{Teorema}


%letras para enumerar
\makeatletter\renewcommand\theenumi{\@alph\c@enumi}\makeatother\renewcommand\labelenumi{\theenumi)}

%m�rgenes
\usepackage[left=2cm,top=2cm,right=2cm]{geometry}

\usepackage{fancyhdr}
\pagestyle{fancy}
\usepackage{enumerate}

%conjuntos N, Q, R, Z, C
\usepackage{dsfont}
\newcommand{\N}{\mathds{N}}
\newcommand{\Q}{\mathds{Q}}
\newcommand{\R}{\mathds{R}}
\newcommand{\Z}{\mathds{Z}}
\newcommand{\C}{\mathds{C}}
\newcommand{\M}{\mathcal{M}}
\newcommand{\Pol}{\mathcal{P}}

%Transformada de Laplace
\renewcommand{\L}{\mathcal{L}}

%Probabilidades
\newcommand{\PP}{\mathbb{P}}
\newcommand{\E}{\mathbb{E}}
\newcommand{\B}{\mathcal{B}}
\newcommand{\Var}{\mathbb{V}\text{ar}}

%l�gica
\newcommand{\ssi}{\Leftrightarrow}

%funciones
\newcommand{\function}[5]{  \begin{array}{rl}
                                #1: #2 &\longrightarrow #3 \\
                                #4 & \longmapsto #1\left(#4\right)= #5
                            \end{array} }

\newcommand{\funcion}[3]{#1: #2 \longrightarrow #3 }
\newcommand{\parteentera}[1]{[#1]}

%%%operadores matematicos
\usepackage{mathtools}
\DeclarePairedDelimiter\abs{\lvert}{\rvert}%
\DeclarePairedDelimiter\norm{\lVert}{\rVert}%
\newcommand{\tr}{\text{tr}}


% Swap the definition of \abs* and \norm*, so that \abs
% and \norm resizes the size of the brackets, and the 
% starred version does not.
\makeatletter
\let\oldabs\abs
\def\abs{\@ifstar{\oldabs}{\oldabs*}}
%
\let\oldnorm\norm
\def\norm{\@ifstar{\oldnorm}{\oldnorm*}}
\makeatother
% % % % % % % % % %
%\providecommand{\abs}[1]{\lvert#1 \rvert}
%\providecommand{\norm}[1]{\lVert#1 \rVert}
%\providecommand{\pin}[2]{\left< #1,#2 \right>} %producto interno

\providecommand{\dpartial}[2]{\frac{\partial #1}{\partial #2}} %derivada parcial

\usepackage{amssymb}
\usepackage{amsmath, amsthm, amsfonts}

%Im�genes
\usepackage{graphicx}
\usepackage{float}



\newcommand{\bit}{\textsc{bit}}
\newcommand{\opt}{\textsc{opt}}


\begin{document}
\fancyhead[L]{Facultad de Ciencias Físicas y Matemáticas}
\fancyhead[R]{Universidad de Chile}

\begin{flushleft}
  \textbf{Online Algorithms and Scheduling}
  \\\textbf{Profesor:} Andreas Wiese.
  \\\textbf{Auxiliar:} Andrés Cristi.
\end{flushleft}


\begin{center}
  \large{\textbf{Clase Auxiliar 2\\ BIT Algorithm for List Update Problem}}
\end{center}

%\begin{flushleft}



\begin{itemize}
\item[\textbf{P1.}] 
Recordemos el problema de List Update o List Accessing. Se tiene una lista $L$ ordenada de $n$ elementos distintos, los que son requeridos
secuencialmente y deben ser servidos inmediatamente. El costo de servir un elemento
es su posición en la lista al momento de ser
requerido. Una vez servido, se puede mover
hacia cualquier posición más adelante sin costo.
Además, se puede hacer intercambio de
elementos adyacentes con costo 1.

Consideremos el algoritmo \bit: Al comienzo
se asocia a cada elemento $x$ de manera
independiente un bit aleatorio
$b(x)$, que es negado cada vez que se accede
a $x$. Si el bit pasa de 0 a 1, entonces $x$
se mueve al frente. Si no, se deja en su lugar.

 El objetivo de este
ejercicio es mostrar que el algoritmo \bit{} es
$1.75$-competitivo. Para ello suponemos que
\opt{} y \bit{} están sirviendo al mismo tiempo
los requerimientos de una instancia $\sigma$.
Enumeramos cada acción $i$, considerando dos
tipos de eventos: ambos algoritmos sirven
un requerimiento, posiblemente con movimientos 
gratis, u \opt{} hace un cambio pagado.
Denotamos por $bit_i$ y $opt_i$ los costos
del evento $i$.

\begin{enumerate}
    \item Muestre que después de servir una secuencia fija
    $\sigma$ de requerimientos, los  bits en
    \bit{} son
    uniformes e independientes.
    
    \item Decimos que $(y,x)$ es una inversión
    si $y$ está antes que $x$ en la lista de
    \bit{}  y al revés en la lista de \opt. Sean
    $\varphi_j$ el número de inversiones
    $(y,x)$ con $b(x)=j$. Denotando por
    \[ \Phi = \varphi_0 + 2\varphi_1 \]
    muestre que al comienzo $\Phi_0=0$.
    
    \item Consideremos el costo amortizado
    \[ a_i = bit_i + \Phi_i - \Phi_{i-1}. \]
    Muestre que si $i$ es un cambio pagado
    de \opt, entonces $\E(a_i)\leq 1.5 \cdot opt_i$.
    
    \item Suponga ahora que $i'$ es un acceso
    al item $x$, que está en la posición $k$
    en \opt. Escriba $\Delta \Phi = A+B+C$, donde
    $A$ es el cambio de potencial generado
    por la creación de inversiones, $B$ es el
    cambio debido a eliminación de inversiones
    y $C$ es el cambio debido a cambio de bit.
    Muestre que $\E(A)\leq \frac{3}{4} (k-1)$.
    
    \item Sea $R$ el número de inversiones de la
    forma $(y,x)$ al momento de acceder a $x$.
    Muestre que $B+C=-R$.
    
    \item Pruebe que $\E(a_{i'}) \leq 1.75 \cdot opt_{i'}$ y concluya que \bit{} es $1.75$-competitivo.
\end{enumerate}

\end{itemize}

\end{document} 
