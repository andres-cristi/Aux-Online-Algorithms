\documentclass[11pt, spanish]{article}
\pagestyle{empty}

%acentos de la forma � en vez de \'a
\usepackage[spanish]{babel}
\usepackage[utf8]{inputenc}

%Enumeración con columnas que se usa con \begin{multicols}{2} \begin{enumerate}
\usepackage{multicol}
\usepackage{multirow}


%Para poder usar begin comment
\usepackage{verbatim}

%Teoremas
\usepackage{amsthm}
\theoremstyle{plain}
\newtheorem{teo}{Teorema}


%letras para enumerar
\makeatletter\renewcommand\theenumi{\@alph\c@enumi}\makeatother\renewcommand\labelenumi{\theenumi)}

%m�rgenes
\usepackage[left=2cm,top=2cm,right=2cm]{geometry}

\usepackage{fancyhdr}
\pagestyle{fancy}
\usepackage{enumerate}

%conjuntos N, Q, R, Z, C
\usepackage{dsfont}
\newcommand{\N}{\mathds{N}}
\newcommand{\Q}{\mathds{Q}}
\newcommand{\R}{\mathds{R}}
\newcommand{\Z}{\mathds{Z}}
\newcommand{\C}{\mathds{C}}
\newcommand{\M}{\mathcal{M}}
\newcommand{\Pol}{\mathcal{P}}

%Transformada de Laplace
\renewcommand{\L}{\mathcal{L}}

%Probabilidades
\newcommand{\PP}{\mathbb{P}}
\newcommand{\E}{\mathbb{E}}
\newcommand{\B}{\mathcal{B}}
\newcommand{\Var}{\mathbb{V}\text{ar}}


%Optimization
\newcommand{\OPT}{\textsc{opt}}
\newcommand{\ALG}{\textsc{alg}}


%l�gica
\newcommand{\ssi}{\Leftrightarrow}

%funciones
\newcommand{\function}[5]{  \begin{array}{rl}
                                #1: #2 &\longrightarrow #3 \\
                                #4 & \longmapsto #1\left(#4\right)= #5
                            \end{array} }

\newcommand{\funcion}[3]{#1: #2 \longrightarrow #3 }
\newcommand{\parteentera}[1]{[#1]}

%%%operadores matematicos
\usepackage{mathtools}
\DeclarePairedDelimiter\abs{\lvert}{\rvert}%
\DeclarePairedDelimiter\norm{\lVert}{\rVert}%
\newcommand{\tr}{\text{tr}}


% Swap the definition of \abs* and \norm*, so that \abs
% and \norm resizes the size of the brackets, and the 
% starred version does not.
\makeatletter
\let\oldabs\abs
\def\abs{\@ifstar{\oldabs}{\oldabs*}}
%
\let\oldnorm\norm
\def\norm{\@ifstar{\oldnorm}{\oldnorm*}}
\makeatother
% % % % % % % % % %
%\providecommand{\abs}[1]{\lvert#1 \rvert}
%\providecommand{\norm}[1]{\lVert#1 \rVert}
%\providecommand{\pin}[2]{\left< #1,#2 \right>} %producto interno

\providecommand{\dpartial}[2]{\frac{\partial #1}{\partial #2}} %derivada parcial

\usepackage{amssymb}
\usepackage{amsmath, amsthm, amsfonts}

%Im�genes
\usepackage{graphicx}
\usepackage{float}
\begin{document}
\fancyhead[L]{Facultad de Ciencias Físicas y Matemáticas}
\fancyhead[R]{Universidad de Chile}

\begin{flushleft}
  \textbf{Tópicos en Matemáticas Discretas III - Online Algorithms and Scheduling}
  \\\textbf{Professor:} Andreas Wiese.
  \\\textbf{Teaching Assistant:} Andrés Cristi.
\end{flushleft}


\begin{center}
  \Large{\textbf{Homework \#1}}
\end{center}

%\begin{flushleft}



\begin{itemize}
  \item[\textbf{P1.}] \textbf{(Ice cream shop)}
    Felipito owns a small ice cream shop with a machine that can
    produce two flavors of ice cream, vanilla and chocolate. 
    At each point in time, the machine is in one of two modes: vanilla mode or chocolate mode. 
    Before changing
     the mode, he has to run the cleaning
    program of the machine which is very long and the cost due to waiting of customers etc. is $\$100$.
    Producing a vanilla ice cream with the machine costs
    $\$10$, producing a chocolate ice cream costs $\$20$. Felipito can produce the ice cream
    manually as well, so he does not have to clean
    the machine, but then the costs are $\$20$ and
    $\$40$, respectively.

    Suppose there is a queue outside the shop of unknown length in which each customer wants exactly one unit of ice cream. 
    Felipito know the flavor that a customer wants only when she is the next in line.
    Suppose the starting mode of the machine is vanilla. What would you recommend 
    Felipito to do in order to minimize his cost? Find a deterministic $2$-competitive algorithm.


\item[\textbf{P2.}] \textbf{(The lost pen problem)}
  You lost your pen below the couch and you want to find it. Since
  there is no much light, the only way is to
  insert your hand in the space between the couch and the floor and
  start moving it until you touch the pen. ¿How could you
  minimize the total distance traveled?

  Let us say that the couch is very long, so we model the position
  of the pen as a point $a\in \R$ and the starting point of the search
  as the origin.
  The \emph{offline} optimal strategy costs $\OPT(a)=|a|$.

  \begin{enumerate}
    \item Is there an strictly competitive algorithm? This is, one such
      that for some constant $c$, $\ALG(a) \leq c\cdot \OPT(a)$, for every
      $a \in \R$ (without the additive constant).
     
    \item Consider the following algorithm for fixed $\beta>1$. You
      start moving to the right $\beta$, then you go back to the origin,
      then you move $\beta^2$ to the left, and continue going back to the
      origin and moving $\beta^i$. What is the asymptotic competitive ratio of this 
      algorithm?

    \item Calculate the value of $\beta$ that minimizes the asymptotic competitive ratio.
  
  \end{enumerate}
  
  \item[\textbf{P3.}] \textbf{(Non-Additive Multislope Ski Rental)}
    You have to buy equipment for a certain task that will take an
    unknown amount of time $T$. There are several types of equipment for performing the task, described by
    the pairs $(b_i,r_i)_{i=0}^k$, with $b_i\leq b_{i+1}$ and $r_i \geq r_{i+1}$.
    If you want to use equipment $i$, you need to pay $b_i$ to purchase it and then you pay $r_i$ per unit time in production cost.
    Assume that $b_0=0$ and $r_k=0$, and that at any time (not necessarily
    integral) you can purchase equipment  $i$ by paying $b_i$, regardless
    of what you have bought so far.
    
\begin{enumerate}
  \item Denote by $\OPT(T)$ the optimal value when
    the task lasts $T$ units of time and find an expression for
    it in terms of $T$ and $(b_i,r_i)_{i=0}^k$. 
    Show that it is a concave function of $T$.
      
    \item Denote by $y(t)$ what you have spent in total at time $t$.
      Assume you are at some timepoint such that $y(t)=2\OPT(t)$. Let
      $u>t$ be a time such that $\OPT(u)= y(t)$ (assume it exists)
      and let $i$ be the most
      expensive item such that $\OPT(u)= b_i + r_i u$. Prove that if you buy
      item $i$ at time $t$ and wait until time $u$ without buying anything
      else, then $y(u)< 2\OPT(u)$. Find an upper bound for $y(t^+)$ (what
      you have spent right after purchasing $i$) to
      show that $y(t') \leq 4 \OPT(t')$ for every $t'\in [t,u]$.
      


    \item Use what you just showed to find a $4$-competitive algorithm. 

  \end{enumerate}




\end{itemize}

\end{document} 
