\documentclass[11pt, spanish]{article}
\pagestyle{empty}

%acentos de la forma � en vez de \'a
\usepackage[spanish]{babel}
\usepackage[utf8]{inputenc}

%Enumeración con columnas que se usa con \begin{multicols}{2} \begin{enumerate}
\usepackage{multicol}
\usepackage{multirow}


%Para poder usar begin comment
\usepackage{verbatim}

%Teoremas
\usepackage{amsthm}
\theoremstyle{plain}
\newtheorem{teo}{Teorema}


%letras para enumerar
\makeatletter\renewcommand\theenumi{\@alph\c@enumi}\makeatother\renewcommand\labelenumi{\theenumi)}

%m�rgenes
\usepackage[left=2cm,top=2cm,right=2cm]{geometry}

\usepackage{fancyhdr}
\pagestyle{fancy}
\usepackage{enumerate}

%conjuntos N, Q, R, Z, C
\usepackage{dsfont}
\newcommand{\N}{\mathds{N}}
\newcommand{\Q}{\mathds{Q}}
\newcommand{\R}{\mathds{R}}
\newcommand{\Z}{\mathds{Z}}
\newcommand{\C}{\mathds{C}}
\newcommand{\M}{\mathcal{M}}
\newcommand{\Pol}{\mathcal{P}}

%Transformada de Laplace
\renewcommand{\L}{\mathcal{L}}

%Probabilidades
\newcommand{\PP}{\mathbb{P}}
\newcommand{\E}{\mathbb{E}}
\newcommand{\B}{\mathcal{B}}
\newcommand{\Var}{\mathbb{V}\text{ar}}


%Optimization
\newcommand{\OPT}{\textsc{opt}}
\newcommand{\ALG}{\textsc{alg}}


%l�gica
\newcommand{\ssi}{\Leftrightarrow}

%funciones
\newcommand{\function}[5]{  \begin{array}{rl}
                                #1: #2 &\longrightarrow #3 \\
                                #4 & \longmapsto #1\left(#4\right)= #5
                            \end{array} }

\newcommand{\funcion}[3]{#1: #2 \longrightarrow #3 }
\newcommand{\parteentera}[1]{[#1]}

%%%operadores matematicos
\usepackage{mathtools}
\DeclarePairedDelimiter\abs{\lvert}{\rvert}%
\DeclarePairedDelimiter\norm{\lVert}{\rVert}%
\newcommand{\tr}{\text{tr}}


% Swap the definition of \abs* and \norm*, so that \abs
% and \norm resizes the size of the brackets, and the 
% starred version does not.
\makeatletter
\let\oldabs\abs
\def\abs{\@ifstar{\oldabs}{\oldabs*}}
%
\let\oldnorm\norm
\def\norm{\@ifstar{\oldnorm}{\oldnorm*}}
\makeatother
% % % % % % % % % %
%\providecommand{\abs}[1]{\lvert#1 \rvert}
%\providecommand{\norm}[1]{\lVert#1 \rVert}
%\providecommand{\pin}[2]{\left< #1,#2 \right>} %producto interno

\providecommand{\dpartial}[2]{\frac{\partial #1}{\partial #2}} %derivada parcial

\usepackage{amssymb}
\usepackage{amsmath, amsthm, amsfonts}

%Im�genes
\usepackage{graphicx}
\usepackage{float}
\begin{document}
\fancyhead[L]{Facultad de Ciencias Físicas y Matemáticas}
\fancyhead[R]{Universidad de Chile}

\begin{flushleft}
  \textbf{Tópicos en Matemáticas Discretas III - Online Algorithms and Scheduling}
  \\\textbf{Professor:} Andreas Wiese.
  \\\textbf{Teaching Assistant:} Andrés Cristi.
\end{flushleft}


\begin{center}
  \Large{\textbf{Homework \#3}}
\end{center}

%\begin{flushleft}



\begin{itemize}
  \item[\textbf{P1.}] 
    \begin{enumerate}
      \item Show that the greedy algorithm for the $k$-server problem
	is not $c$-competitive for any constant $c$.
      \item A space $(X,d)$ is called asymmetric if $d:X^2\rightarrow [0,+\infty]$
	satisfies the conditions
	\begin{enumerate}[i.]
	  \item $d(x,x)=0, \forall x\in X$
	  \item $d(x,y)\leq d(x,z)+d(z,y), \forall x,y,z\in X$
	\end{enumerate}
	but it is not symmetric. Show that
	the competitive ratio for $k$-server for such spaces cannot be bounded
	by a function of $k$.
    \end{enumerate}

  
  \item[\textbf{P2.}] Consider a uniform MTS on $n$ states, i.e. a space $(X,d)$ such
    that $d(x,y)=1$ for every pair $x,y\in X$; in which the tasks are functions $\tau:X\rightarrow \{0,1\}$.
    \begin{enumerate}
      \item Find a random instance on $k$ tasks such that the expected cost of any
	deterministic algorithm is $k/n$.
      \item Let $c(k)$ be the expected cost of the optimal offline algorithm.
	Prove that $\lim_k \frac{c(k)}{k}= \frac{1}{nH_n}$, where $H_n$ is 
	the $n$-th harmonic number. It will be helpful to remember the following result: let $(X_i)_{i\in\N}$ be a sequence of positive and iid random
	varibles with finite expectation, and define $S_n= \sum_{i=1}^n X_i$.
	We call $N(t)= \min \{ n\in \N: S_n\geq t \}$ a renewal process, and the
	following limit holds $\lim_{t\rightarrow\infty} \frac{\E\big(N(t)\big)}{t}= \frac{1}{\E(X_1)}$.
	
      \item Conclude that no random algorithm can be better than $H_n$-competitive.
      \item For a state $s$ and a sequence of tasks $\hat{\tau}$, define
	$\rho(s,\hat{\tau})= w(s,\hat{\tau}) - \min_x w(x,\hat{\tau})$, where
	$w$ is the work function of the MTS. Consider $\rho(\cdot,\hat{\tau})$ as
	a point in $\R^X$, and describe its possible values.
      \item Consider now $\rho$ as a description of the system after a secuence
	of tasks. Find a reasonable random algorithm that moves to a random
	state given by a distribution that depends only on $\rho$, that is
	$H_n$-competitive. \textbf{Hint:} Consider the potential $\Phi=H_m$, where
	$m= | \arg\min_{x\in X} \rho(x,\hat{\tau}) |$.
    \end{enumerate}

  \item[\textbf{P3.}] Use the DC-TREE algorithm to construct an algorithm for paging problem with competitive ratio $k$. What is the interpretation of this algorithm in paging, ignoring $k$-sever?

  \item[\textbf{P4.}] Consider the class of metric spaces $(\mathcal{M},d)$, where $\mathcal{M}$

  is a convex subset of a vector space and $d$ satisfies that for every three different points
  $x,y,r\in \mathcal{M}$ such that $y= \lambda x + (1-\lambda)r$, for some $\lambda\in (0,1)$,
  the following implication is true:
  \begin{align*}
    \big[ d(p,x) \leq d(p,r) \Rightarrow d(p,y) \leq d(p,r)  \big], \forall p\in X.
  \end{align*}
  \begin{enumerate}
    \item Prove that every euclidean space is in this class of
  	spaces.
      \item Prove that the algorithm SC$_{1/2}$ we studied in the TA class is 3-competitive
	for every space in this class.
    \end{enumerate}

  \item[\textbf{P5.}] Consider the following 2-server algorithm. After serving each request,
    label the server at the request as $s_1$ and the other as $s_2$ (if both are in the same
    point, label arbitrarily). Consider the next request $r$ and set $b= d(s_1,r)$. If
    $d(s_2,r)< 3b$, serve $r$ with $s_2$. Otherwise, serve it with $s_1$ and move $s_2$
    a distance $3b$ towards $r$. Prove that this algorithm is $O(1)$-competitive in any
    Euclidean space.


\end{itemize}

\end{document} 
