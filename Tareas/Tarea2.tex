\documentclass[11pt, spanish]{article}
\pagestyle{empty}

%acentos de la forma � en vez de \'a
\usepackage[spanish]{babel}
\usepackage[utf8]{inputenc}

%Enumeración con columnas que se usa con \begin{multicols}{2} \begin{enumerate}
\usepackage{multicol}
\usepackage{multirow}


%Para poder usar begin comment
\usepackage{verbatim}

%Teoremas
\usepackage{amsthm}
\theoremstyle{plain}
\newtheorem{teo}{Teorema}


%letras para enumerar
\makeatletter\renewcommand\theenumi{\@alph\c@enumi}\makeatother\renewcommand\labelenumi{\theenumi)}

%m�rgenes
\usepackage[left=2cm,top=2cm,right=2cm]{geometry}

\usepackage{fancyhdr}
\pagestyle{fancy}
\usepackage{enumerate}

%conjuntos N, Q, R, Z, C
\usepackage{dsfont}
\newcommand{\N}{\mathds{N}}
\newcommand{\Q}{\mathds{Q}}
\newcommand{\R}{\mathds{R}}
\newcommand{\Z}{\mathds{Z}}
\newcommand{\C}{\mathds{C}}
\newcommand{\M}{\mathcal{M}}
\newcommand{\Pol}{\mathcal{P}}

%Transformada de Laplace
\renewcommand{\L}{\mathcal{L}}

%Probabilidades
\newcommand{\PP}{\mathbb{P}}
\newcommand{\E}{\mathbb{E}}
\newcommand{\B}{\mathcal{B}}
\newcommand{\Var}{\mathbb{V}\text{ar}}


%Optimization
\newcommand{\OPT}{\textsc{opt}}
\newcommand{\ALG}{\textsc{alg}}


%l�gica
\newcommand{\ssi}{\Leftrightarrow}

%funciones
\newcommand{\function}[5]{  \begin{array}{rl}
                                #1: #2 &\longrightarrow #3 \\
                                #4 & \longmapsto #1\left(#4\right)= #5
                            \end{array} }

\newcommand{\funcion}[3]{#1: #2 \longrightarrow #3 }
\newcommand{\parteentera}[1]{[#1]}

%%%operadores matematicos
\usepackage{mathtools}
\DeclarePairedDelimiter\abs{\lvert}{\rvert}%
\DeclarePairedDelimiter\norm{\lVert}{\rVert}%
\newcommand{\tr}{\text{tr}}


% Swap the definition of \abs* and \norm*, so that \abs
% and \norm resizes the size of the brackets, and the 
% starred version does not.
\makeatletter
\let\oldabs\abs
\def\abs{\@ifstar{\oldabs}{\oldabs*}}
%
\let\oldnorm\norm
\def\norm{\@ifstar{\oldnorm}{\oldnorm*}}
\makeatother
% % % % % % % % % %
%\providecommand{\abs}[1]{\lvert#1 \rvert}
%\providecommand{\norm}[1]{\lVert#1 \rVert}
%\providecommand{\pin}[2]{\left< #1,#2 \right>} %producto interno

\providecommand{\dpartial}[2]{\frac{\partial #1}{\partial #2}} %derivada parcial

\usepackage{amssymb}
\usepackage{amsmath, amsthm, amsfonts}

%Im�genes
\usepackage{graphicx}
\usepackage{float}
\begin{document}
\fancyhead[L]{Facultad de Ciencias Físicas y Matemáticas}
\fancyhead[R]{Universidad de Chile}

\begin{flushleft}
  \textbf{Tópicos en Matemáticas Discretas III - Online Algorithms and Scheduling}
  \\\textbf{Professor:} Andreas Wiese.
  \\\textbf{Teaching Assistant:} Andrés Cristi.
\end{flushleft}


\begin{center}
  \Large{\textbf{Homework \#2}}
\end{center}

%\begin{flushleft}



\begin{itemize}
\item[\textbf{P1.}] 
  \begin{enumerate}
    \item Formulate the Ski Rental Problem as a Metrical Task System.
    \item Formulate the list accessing problem as a Metrical Task System.
    \item Consider the greedy algorithm for an MTS, this is, one that
      from state $s_i$ moves to $s_{i+1}\in \arg\min_x d(s_i,x) + \tau_{i+1}(x)$
      to process task $\tau_{i+1}$. Show that this algorithm has an
      unbounded competitive ratio.
  \end{enumerate}
  
  \item[\textbf{P2.}]  Using Yao's Principle find a lower bound of at least $1.1$ for
    the competitivity ratio of any randomized algorithm for the list 
    accessing problem.


      \item[\textbf{P3.}] A stack is a Last-in-First-Out memory/buffer $S$, on which
	the following operations are defined:
	\begin{itemize}
	  \item \textsc{Push}$(S,x)$ adds the object $x$ on top of the stack,
	    with cost 1.
	  \item \textsc{Pop}$(S)$ returns the topmost object of the stack
	    and removes it, with cost 1.
	  \item \textsc{Multipop}$(S,k)$ returns the topmost $k$ objects with cost $k$.
	\end{itemize}
	We analyze the cost of processing a sequence of $n$ stack operations.
	\begin{enumerate}
	  \item Apply the usual worst case analysis to obtain a bound of $O(n^2)$ in
	    the processing time, when begining with an empty stack.
	  \item Let $c_i$ be the cost of the $i$-th operation and $S_i$ the stack
	    after the $i$-th operation. For some potential function $\Phi$, define
	    the amortized cost
	    \begin{align*}
	      a_i= c_i + \Phi(S_i) - \Phi(S_{i-1})
	    \end{align*}
	    and show that the total cost is at most $\Phi(S_0) + \sum_{i=1}^n a_i$.
	  \item Use the potential $\Phi(S)= |S|$ to show a bound of $O(n)$ in the
	    processing time (assume that whenever a \textsc{Pop} or \textsc{Multipop}
	    operation is done there are sufficiently many elements on the stack).
	\end{enumerate}

      \item[\textbf{P4.}] A \textit{tight example} for a $c$-competitive algorithm
	for a minimization problem is a family of instances $(I_\varepsilon)_{\varepsilon>0}$
	 such that
	\begin{align*}
	  \text{ALG}(I_\varepsilon) \geq (c- \varepsilon) \text{OPT}(I_\varepsilon).
	\end{align*}
	Find a \textit{tight example} for the traversal algorithm for Metrical Task
	Systems with $N$ states.


\end{itemize}

\end{document} 
