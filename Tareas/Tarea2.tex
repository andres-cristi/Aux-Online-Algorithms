\documentclass[11pt, spanish]{article}
\pagestyle{empty}

%acentos de la forma � en vez de \'a
\usepackage[spanish]{babel}
\usepackage[utf8]{inputenc}

%Enumeración con columnas que se usa con \begin{multicols}{2} \begin{enumerate}
\usepackage{multicol}
\usepackage{multirow}


%Para poder usar begin comment
\usepackage{verbatim}

%Teoremas
\usepackage{amsthm}
\theoremstyle{plain}
\newtheorem{teo}{Teorema}


%letras para enumerar
\makeatletter\renewcommand\theenumi{\@alph\c@enumi}\makeatother\renewcommand\labelenumi{\theenumi)}

%m�rgenes
\usepackage[left=2cm,top=2cm,right=2cm]{geometry}

\usepackage{fancyhdr}
\pagestyle{fancy}
\usepackage{enumerate}

%conjuntos N, Q, R, Z, C
\usepackage{dsfont}
\newcommand{\N}{\mathds{N}}
\newcommand{\Q}{\mathds{Q}}
\newcommand{\R}{\mathds{R}}
\newcommand{\Z}{\mathds{Z}}
\newcommand{\C}{\mathds{C}}
\newcommand{\M}{\mathcal{M}}
\newcommand{\Pol}{\mathcal{P}}

%Transformada de Laplace
\renewcommand{\L}{\mathcal{L}}

%Probabilidades
\newcommand{\PP}{\mathbb{P}}
\newcommand{\E}{\mathbb{E}}
\newcommand{\B}{\mathcal{B}}
\newcommand{\Var}{\mathbb{V}\text{ar}}


%Optimization
\newcommand{\OPT}{\textsc{opt}}
\newcommand{\ALG}{\textsc{alg}}


%l�gica
\newcommand{\ssi}{\Leftrightarrow}

%funciones
\newcommand{\function}[5]{  \begin{array}{rl}
                                #1: #2 &\longrightarrow #3 \\
                                #4 & \longmapsto #1\left(#4\right)= #5
                            \end{array} }

\newcommand{\funcion}[3]{#1: #2 \longrightarrow #3 }
\newcommand{\parteentera}[1]{[#1]}

%%%operadores matematicos
\usepackage{mathtools}
\DeclarePairedDelimiter\abs{\lvert}{\rvert}%
\DeclarePairedDelimiter\norm{\lVert}{\rVert}%
\newcommand{\tr}{\text{tr}}


% Swap the definition of \abs* and \norm*, so that \abs
% and \norm resizes the size of the brackets, and the 
% starred version does not.
\makeatletter
\let\oldabs\abs
\def\abs{\@ifstar{\oldabs}{\oldabs*}}
%
\let\oldnorm\norm
\def\norm{\@ifstar{\oldnorm}{\oldnorm*}}
\makeatother
% % % % % % % % % %
%\providecommand{\abs}[1]{\lvert#1 \rvert}
%\providecommand{\norm}[1]{\lVert#1 \rVert}
%\providecommand{\pin}[2]{\left< #1,#2 \right>} %producto interno

\providecommand{\dpartial}[2]{\frac{\partial #1}{\partial #2}} %derivada parcial

\usepackage{amssymb}
\usepackage{amsmath, amsthm, amsfonts}

%Im�genes
\usepackage{graphicx}
\usepackage{float}
\begin{document}
\fancyhead[L]{Facultad de Ciencias Físicas y Matemáticas}
\fancyhead[R]{Universidad de Chile}

\begin{flushleft}
  \textbf{Tópicos en Matemáticas Discretas III - Online Algorithms and Scheduling}
  \\\textbf{Professor:} Andreas Wiese.
  \\\textbf{Teaching Assistant:} Andrés Cristi.
\end{flushleft}


\begin{center}
  \Large{\textbf{Homework \#2}}
\end{center}

%\begin{flushleft}



\begin{itemize}
  \item[\textbf{P1.}]
    \begin{enumerate}
      \item Consider an arbitrary two-person zero-sum game
    	defined by the $(n\times m)$-matrix $M = (m_{ij})$. Let
    	\begin{align*}
      		V_R &= \max_i \min_j m_{ij} \\
      		V_C &= \min_j \max_i m_{ij}
    	\end{align*}
	Show that $V_R\leq V_C$.
      \item Give an example in which $V_R< V_C$.

    \end{enumerate}
    


\item[\textbf{P2.}] 
  \begin{enumerate}
    \item Formulate the Ski Rental Problem as a Metrical Task System.
    \item Formulate the list accessing problem as a Metrical Task System.
    \item Consider the greedy algorithm for an MTS, this is, one that
      from state $s_i$ moves to $s_{i+1}\in \arg\min_x d(s_i,x) + \tau_{i+1}(x)$
      to process task $\tau_{i+1}$. Show that this algorithm is not
      competitive at all.
  \end{enumerate}
  
  \item[\textbf{P3.}]  In this exercise we compute a lower bound for
    the competitivity ratio of any randomized algorithm for the list
    accessing problem. For that purpose, consider an instance where
    the requests are drawn uniformly at random from $\{1,\dots n\}$.
    
	\begin{enumerate}
	  \item Calculate the expected cost of serving one request
	    for a deterministic algorithm.
	  \item Calculate the expected optimal offline cost if $n=2$ and
	    there are only 3 requests.
	  \item Conclude a lower bound of $12/11$ for the (asymptotic) 
	    competitivity of any randomized algorithm. 
	  \item Get a strictly better bound.

	\end{enumerate}




\end{itemize}

\end{document} 
