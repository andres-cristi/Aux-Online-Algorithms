\documentclass[11pt, spanish]{article}
\pagestyle{empty}

%acentos de la forma � en vez de \'a
\usepackage[spanish]{babel}
\usepackage[utf8]{inputenc}

%Enumeración con columnas que se usa con \begin{multicols}{2} \begin{enumerate}
\usepackage{multicol}
\usepackage{multirow}


%Para poder usar begin comment
\usepackage{verbatim}

%Teoremas
\usepackage{amsthm}
\theoremstyle{plain}
\newtheorem{teo}{Teorema}


%letras para enumerar
\makeatletter\renewcommand\theenumi{\@alph\c@enumi}\makeatother\renewcommand\labelenumi{\theenumi)}

%m�rgenes
\usepackage[left=2cm,top=2cm,right=2cm]{geometry}

\usepackage{fancyhdr}
\pagestyle{fancy}
\usepackage{enumerate}

%conjuntos N, Q, R, Z, C
\usepackage{dsfont}
\newcommand{\N}{\mathds{N}}
\newcommand{\Q}{\mathds{Q}}
\newcommand{\R}{\mathds{R}}
\newcommand{\Z}{\mathds{Z}}
\newcommand{\C}{\mathds{C}}
\newcommand{\M}{\mathcal{M}}
\newcommand{\Pol}{\mathcal{P}}

%Transformada de Laplace
\renewcommand{\L}{\mathcal{L}}

%Probabilidades
\newcommand{\PP}{\mathbb{P}}
\newcommand{\E}{\mathbb{E}}
\newcommand{\B}{\mathcal{B}}
\newcommand{\Var}{\mathbb{V}\text{ar}}

%l�gica
\newcommand{\ssi}{\Leftrightarrow}

%funciones
\newcommand{\function}[5]{  \begin{array}{rl}
                                #1: #2 &\longrightarrow #3 \\
                                #4 & \longmapsto #1\left(#4\right)= #5
                            \end{array} }

\newcommand{\funcion}[3]{#1: #2 \longrightarrow #3 }
\newcommand{\parteentera}[1]{[#1]}

%%%operadores matematicos
\usepackage{mathtools}
\DeclarePairedDelimiter\abs{\lvert}{\rvert}%
\DeclarePairedDelimiter\norm{\lVert}{\rVert}%
\newcommand{\tr}{\text{tr}}


% Swap the definition of \abs* and \norm*, so that \abs
% and \norm resizes the size of the brackets, and the 
% starred version does not.
\makeatletter
\let\oldabs\abs
\def\abs{\@ifstar{\oldabs}{\oldabs*}}
%
\let\oldnorm\norm
\def\norm{\@ifstar{\oldnorm}{\oldnorm*}}
\makeatother
% % % % % % % % % %
%\providecommand{\abs}[1]{\lvert#1 \rvert}
%\providecommand{\norm}[1]{\lVert#1 \rVert}
%\providecommand{\pin}[2]{\left< #1,#2 \right>} %producto interno

\providecommand{\dpartial}[2]{\frac{\partial #1}{\partial #2}} %derivada parcial

\usepackage{amssymb}
\usepackage{amsmath, amsthm, amsfonts}

%Im�genes
\usepackage{graphicx}
\usepackage{float}



\newcommand{\bit}{\textsc{bit}}
\newcommand{\opt}{\textsc{opt}}
\newcommand{\SL}{\text{slack}}


\begin{document}
\fancyhead[L]{Facultad de Ciencias Físicas y Matemáticas}
\fancyhead[R]{Universidad de Chile}

\begin{flushleft}
  \textbf{Online Algorithms and Scheduling}
  \\\textbf{Profesor:} Andreas Wiese.
  \\\textbf{Auxiliar:} Andrés Cristi.
\end{flushleft}


\begin{center}
  \large{\textbf{Clase Auxiliar 6\\ 23 de Abril }}
\end{center}

%\begin{flushleft}



\begin{itemize}
  \item[\textbf{P1.}] \textbf{Slack Coverage Algorithm}
  
  

	  En el espacio $(\R^2,d)$, con $d$ la distancia Euclideana, definimos
	  \begin{align*}
		 \SL(x,y,r) = d(y,x) + d(x,r) - d(y,r) 
	  \end{align*}
	  que por la desigualdad triangular es siempre un valor no negativo.

	  Para el problema 2-\textit{server} consideremos el algoritmo
	  Slack-Coverage$(\gamma)$ (que abreviamos SC$_\gamma$), con
	  $\gamma\in [0,1]$. Este algoritmo frente a un request $r$
	  si $x$ es el servidor más cercano, mueve el otro servidor, $y$,
	  hacia el servidor $x$ una distancia $\gamma\cdot \SL(x,y,r)$ y luego
	  sirve $r$ con $x$.

	  Consideremos el potencial
	  \begin{align*}
		  \Phi = a M_{\min} + b\cdot d(x,y)
	  \end{align*}
	  donde $M_{\min}$ es el valor del \textit{matching} mínimo
	  entre los servidores de SC$_{\gamma}$ y OPT, y $a,b$ son 
	  parámetros positivos.

	  \begin{enumerate}
		  \item Pruebe que cuando OPT sirve un request, $\Phi$ 
			  aumenta en a lo más $a\cdot c_{\text{OPT}}$, con
			  $c_{\text{OPT}}$ la distancia que mueve OPT.
		  \item Denotando $x_0, y_0$ a las posiciones de los 
			  servidores de SC$_\gamma$ antes de servir $r$,
			  pruebe que
			  \begin{align*}
				  \Delta d(x,y) \leq d(y_0,r) + d(x_0,y_0).
			  \end{align*}
		  \item Sean $s_1$ y $s_2$ los servidores de OPT, donde $s_1$ es
			  el que acaba de servir $r$. Suponga que antes de que SC$_{\gamma}$
			  mueva, $x$ estaba emparejado con $s_1$ en $M_{\min}$.
			  Pruebe que
			  \begin{align*}
				  \Delta \Phi\leq a\big( \gamma\cdot \SL(x,y,r)-d(x,r) \big)
				  +b \big( d(y,r) - d(x,y) \big).
			  \end{align*}
		  \item En el caso en que $y$ era el que estaba emparejado con $s_1$, pruebe
			  que
			  \begin{align*}
				  \Delta \Phi \leq a\big( d(x,y) - \gamma \cdot\SL(x,y,r) - d(y,r) \big)
				  + b\big( d(y,r) - d(x,y) \big)
			  \end{align*}
		  \item Encuentre las condiciones sobre $(a,b,\gamma)$ para que 
			  SC$_{\gamma}$ sea $a$-competitivo.
		  \item Pruebe que SC$_{\gamma}$ es 3-competitivo y que este es
			  el mejor factor que permite este m\'etodo.
	  \end{enumerate}

    
    
\end{itemize}

\end{document} 
