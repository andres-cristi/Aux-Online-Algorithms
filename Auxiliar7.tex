\documentclass[11pt, spanish]{article}
\pagestyle{empty}

%acentos de la forma � en vez de \'a
\usepackage[spanish]{babel}
\usepackage[utf8]{inputenc}

%Enumeración con columnas que se usa con \begin{multicols}{2} \begin{enumerate}
\usepackage{multicol}
\usepackage{multirow}


%Para poder usar begin comment
\usepackage{verbatim}

%Teoremas
\usepackage{amsthm}
\theoremstyle{plain}
\newtheorem{teo}{Teorema}


%letras para enumerar
\makeatletter\renewcommand\theenumi{\@alph\c@enumi}\makeatother\renewcommand\labelenumi{\theenumi)}

%m�rgenes
\usepackage[left=2cm,top=2cm,right=2cm]{geometry}

\usepackage{fancyhdr}
\pagestyle{fancy}
\usepackage{enumerate}

%conjuntos N, Q, R, Z, C
\usepackage{dsfont}
\newcommand{\N}{\mathds{N}}
\newcommand{\Q}{\mathds{Q}}
\newcommand{\R}{\mathds{R}}
\newcommand{\Z}{\mathds{Z}}
\newcommand{\C}{\mathds{C}}
\newcommand{\M}{\mathcal{M}}
\newcommand{\Pol}{\mathcal{P}}

%Transformada de Laplace
\renewcommand{\L}{\mathcal{L}}

%Probabilidades
\newcommand{\PP}{\mathbb{P}}
\newcommand{\E}{\mathbb{E}}
\newcommand{\B}{\mathcal{B}}
\newcommand{\Var}{\mathbb{V}\text{ar}}

%l�gica
\newcommand{\ssi}{\Leftrightarrow}

%funciones
\newcommand{\function}[5]{  \begin{array}{rl}
                                #1: #2 &\longrightarrow #3 \\
                                #4 & \longmapsto #1\left(#4\right)= #5
                            \end{array} }

\newcommand{\funcion}[3]{#1: #2 \longrightarrow #3 }
\newcommand{\parteentera}[1]{[#1]}

%%%operadores matematicos
\usepackage{mathtools}
\DeclarePairedDelimiter\abs{\lvert}{\rvert}%
\DeclarePairedDelimiter\norm{\lVert}{\rVert}%
\newcommand{\tr}{\text{tr}}


% Swap the definition of \abs* and \norm*, so that \abs
% and \norm resizes the size of the brackets, and the 
% starred version does not.
\makeatletter
\let\oldabs\abs
\def\abs{\@ifstar{\oldabs}{\oldabs*}}
%
\let\oldnorm\norm
\def\norm{\@ifstar{\oldnorm}{\oldnorm*}}
\makeatother
% % % % % % % % % %
%\providecommand{\abs}[1]{\lvert#1 \rvert}
%\providecommand{\norm}[1]{\lVert#1 \rVert}
%\providecommand{\pin}[2]{\left< #1,#2 \right>} %producto interno

\providecommand{\dpartial}[2]{\frac{\partial #1}{\partial #2}} %derivada parcial

\usepackage{amssymb}
\usepackage{amsmath, amsthm, amsfonts}

%Im�genes
\usepackage{graphicx}
\usepackage{float}



\newcommand{\bit}{\textsc{bit}}
\newcommand{\opt}{\textsc{opt}}
\newcommand{\SL}{\text{slack}}


\begin{document}
\fancyhead[L]{Facultad de Ciencias Físicas y Matemáticas}
\fancyhead[R]{Universidad de Chile}

\begin{flushleft}
  \textbf{Online Algorithms and Scheduling}
  \\\textbf{Profesor:} Andreas Wiese.
  \\\textbf{Auxiliar:} Andrés Cristi.
\end{flushleft}


\begin{center}
  \large{\textbf{Clase Auxiliar 7\\ 07 de Mayo }}
\end{center}

%\begin{flushleft}



\begin{itemize}
	\item[\textbf{P1.}] Si $w$ es la \textit{work function}
		actual, $r$ es el siguiente request y $w'$ es
		la actualización de $w$ con $r$, decimos que
		la configuración $A$ es máximo con respecto a
		$w$ si maximiza $\max_X\{ w'(X)-w(X)\}$ y que es
		mínimo para $r$ con respecto a $w$ si minimiza
		$\min_X\{w(X) - \sum_{x\in X} d(x,r) \}$. En cátedra
		probamos que si $A$ es mínimo entonces tambi\'en
		es máximo. Considere el espacio m\'etrico dado
		por los puntos en un círculo y la distancia sobre
		\'este.
		\begin{enumerate}
			\item Muestre que si $\bar{r}_i$ es el punto
				opuesto en el círculo a $r_i$, entonces
				la configuración $\bar{r}_i^k$ ($k$
				copias de $\bar{r}_i$) es mínimo
				para $r_i$ con respecto a $w_{i-1}$.
			\item Pruebe que las siguientes desigualdades son
				equivalentes (para ciertas $C, C'$ constantes
				que no dependen de la instancia)
				\begin{align}
					\sum_{i=1}^n \max_X\{ w_i(X)-w_{i-1}
				(X)\}
				&\leq (c+1) \min_Y\{w_n(Y)\} + C\\
				\sum_{i=1}^n \left( w_i(\bar{r}_i^k)
				-w_{i-1}(\bar{r}_i^k) \right)
				&\leq (c+1) w_n(\bar{r}_n^k) + C'
				\end{align}
			\item Pruebe que si $i<j$, entonces
				$w_i(\bar{r}_i^k)\leq w_{j-1} + kd(\bar{r}_i,
				\bar{r}_j)$.
			\item Sea $OPT_h(\sigma)$ el óptimo offline
				al servir $\sigma$ con $h$ servidores.
				Pruebe que hay exactamente $h$ requests
				$i$ tales que el servidor que sirve $i$
				no se mueve más hasta el final de la
				instancia.
			\item Pruebe que existe una constante $C$ que no
				depende de $\sigma$, tal que para todo
				$h$
				\begin{align}
					\sum_{i=1}^n w_i(\bar{r}_i^k)\leq
					\sum_{i=1}^n w_{i-1}(\bar{r}_i^k)
					+k\cdot OPT_h(\sigma)
					+h\cdot OPT_k(\sigma) + C
				\end{align}
			\item Concluya que el WFA$_k$ es $2k-1$ competitivo
				en el círculo,
				y muestre una cota más ajustada para el
				caso en que se compara contra un óptimo
				offline con menos servidores.
			\item Qu\'e tan necesario es que el espacio
				fuera un círculo? Considere un espacio
				en el que para todo punto $a$ existe
				un punto opuesto $\bar{a}$ tal que
				para todo punto $b$, $d(a,b)+d(b,\bar{a})=
				\Delta$.
		\end{enumerate} 
    
    
\end{itemize}

\end{document} 
